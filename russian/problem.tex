\begin{problem}{Глеб и медиана}{стандартный ввод}{стандартный вывод}{5 секунд}{512 мегабайт}

Глеб устал от побитового исключающего <<или>>  и решил, что пора найти новую интересную функцию. Его выбор пал на медиану. Напомним, медианой массива называется число, которое окажется посередине, если массив упорядочить по возрастанию. В рамках этой задачи для массивов чётной длины положим медиану равной левому из двух центральных в отсортированном порядке элементов.

Для некоторого числа $m$ назовём $m$-разбиением массива такое его разбиение на непересекающиеся отрезки, что на каждом из этих отрезков медиана больше либо равна $m$. Вам дан массив $a$ длины $n$ и $q$ запросов двух видов:

\begin{enumerate}
\item присвоить элементу с индексом $i$ значение $x$;
\item найти наибольшее число $k$ такое, что для подотрезка массива с индексами от $l$ до $r$ существует $m$-разбиение на $k$ отрезков. 
\end{enumerate}

\InputFile
В первой строке дается число $n$ ($1 \le n \le 2 \cdot 10^5$) - размер массива. В следующей строке вводятся $n$ чисел $a_{i}$ ($1 \le a_{i} \le 10^9$) - элементы массива, на следующей строке вводится число $q$ ($1 \le q \le 2 \cdot 10^5$) - количество запросов. В следующих $q$ строках даются запросы, каждый в одном из следующих форматов:
\begin{itemize}
\item $1$ $i$ $x$ ---~ запрос 1 типа ($1 \le i \le n, 1 \le x \le 10^9$);
\item $2$ $m$ $l$ $r$ ---~ запрос 2 типа ($1 \le m \le 10^9, 1 \le l \le r \le n$).
\end{itemize}

\OutputFile
Для каждого запроса второго типа в отдельной строке выведите ответ на запрос. В случае если для отрезка не существует никакого $m$-разбиения, выведите $0$.

\Scoring
В задаче присутствуют 6 групп:
\begin{enumerate}
\item $n \le 5, q \le 5$, такие решения будут набирать не менее 10\% баллов
\item $n \le 100, q \le 100$, такие решения будут набирать не менее 20\% баллов
\item $n \le 10000, q \le 10000$, такие решения будут набирать не менее 30\% баллов
\item $m$ - одно и тоже для всех запросов, такие решения будут набирать не менее 25\% баллов
\item нет запросов изменения, такие решения будут набирать не менее 35\% баллов
\item Ограничения, как и в задаче
\end{enumerate}

\Example

\begin{example}
\exmpfile{example.01}{example.01.a}%
\end{example}

\end{problem}

